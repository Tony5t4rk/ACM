\documentclass[a4paper,11pt]{article}
\usepackage{zh_CN-Adobefonts_external} % Simplified Chinese Support using external fonts (./fonts/zh_CN-Adobe/)
\usepackage{fancyhdr}  % 页眉页脚
\usepackage{minted}    % 代码高亮
\usepackage[colorlinks]{hyperref}  % 目录可跳转
\setlength{\headheight}{15pt}

% 定义页眉页脚
\pagestyle{fancy}
\fancyhf{}
\fancyhead[C]{Algorithm Library by Liu Yang}
\lfoot{}
\cfoot{\thepage}
\rfoot{}

\author{Liu Yang}
\title{Algorithm Library}

\begin{document} 
\maketitle % 封面
\newpage % 换页
\tableofcontents % 目录
\newpage
\section{字符串}
\subsection{KMP}
\inputminted[breaklines]{c++}{字符串/KMP.cpp}


\section{动态规划}
\subsection{最长不下降子序列}
\inputminted[breaklines]{c++}{动态规划/最长不下降子序列_LIS.cpp}
\subsection{最长公共子序列}
\inputminted[breaklines]{c++}{动态规划/最长公共子序列_LCS.cpp}
\subsection{背包}
\inputminted[breaklines]{c++}{动态规划/背包.cpp}


\section{数据结构}
\subsection{树状数组}
\inputminted[breaklines]{c++}{数据结构/树状数组.cpp}
\subsection{线段树}
\subsubsection{线段树-Array}
\inputminted[breaklines]{c++}{数据结构/线段树.cpp}
\subsubsection{线段树-Struct}
\inputminted[breaklines]{c++}{数据结构/线段树-Struct.cpp}
\subsection{伸展树}
\subsubsection{Splay-维护二叉查找树}
\inputminted[breaklines]{c++}{数据结构/Splay-维护二叉查找树.cpp}
\subsubsection{Splay-维护数列}
\inputminted[breaklines]{c++}{数据结构/Splay-维护数列.cpp}
\subsection{字典树}
\inputminted[breaklines]{c++}{数据结构/字典树_Trie-链表.cpp}
\subsection{Dfs序}
\inputminted[breaklines]{c++}{数据结构/Dfs序.cpp}
\subsection{最近公共祖先}
\subsubsection{在线LCA}
\inputminted[breaklines]{c++}{数据结构/在线LCA-DFS+ST.cpp}
\subsubsection{离线LCA}
\inputminted[breaklines]{c++}{数据结构/离线LCA-Tarjan.cpp}


\section{图论}
\subsection{最小生成树}
\subsubsection{Prim-邻接表}
\inputminted[breaklines]{c++}{图论/Prim-邻接表.cpp}
\subsubsection{Kruskal}
\inputminted[breaklines]{c++}{图论/Kruskal.cpp}
\subsection{最短路}
\subsubsection{Bellman-Ford(判负环)}
\inputminted[breaklines]{c++}{图论/Bellman-Ford.cpp}
\subsubsection{Dijkstra-邻接表}
\inputminted[breaklines]{c++}{图论/Dijkstra-邻接表.cpp}
\subsubsection{Dijkstra-堆优化-邻接表}
\inputminted[breaklines]{c++}{图论/Dijkstra-邻接表-堆优化.cpp}
\subsubsection{Dijkstra-堆优化-链式前向星}
\inputminted[breaklines]{c++}{图论/Dijkstra-链式前向星-堆优化.cpp}
\subsubsection{Spfa-邻接表}
\inputminted[breaklines]{c++}{图论/SPFA-邻接表.cpp}
\subsubsection{Floyd}
\inputminted[breaklines]{c++}{图论/Floyd.cpp}
\subsection{第K短路}
\subsubsection{A*算法-链式前向星}
\inputminted[breaklines]{c++}{图论/AStar.cpp}
\subsection{二分图匹配}
\subsubsection{匈牙利算法-链式前向星}
\inputminted[breaklines]{c++}{图论/匈牙利算法-链式前向星.cpp}
\subsection{最大流}
\subsubsection{Ford-Fulkerson-邻接矩阵}
\inputminted[breaklines]{c++}{图论/Ford-Fulkerson-邻接矩阵.cpp}
\subsubsection{Dinic-邻接矩阵}
\inputminted[breaklines]{c++}{图论/Dinic-邻接矩阵.cpp}
\subsubsection{Dinic-链式前向星}
\inputminted[breaklines]{c++}{图论/Dinic-链式前向星.cpp}
\subsection{费用流}
\subsubsection{最小费用最大流-Spfa}
\inputminted[breaklines]{c++}{图论/最小费用最大流-SPFA.cpp}


\section{计算几何}
\subsubsection{凸包}
\inputminted[breaklines]{c++}{计算几何/凸包.cpp}
\section{数论}
\subsection{素数}
\inputminted[breaklines]{c++}{数论/素数-筛法.cpp}
\subsection{母函数}
\inputminted[breaklines]{c++}{数论/母函数.cpp}
\subsection{快速乘+快速幂}
\inputminted[breaklines]{c++}{数论/快速幂+快速乘.cpp}
\subsection{卡特兰}
\inputminted[breaklines]{c++}{数论/卡特兰数列.cpp}
\subsection{斯特林}
\inputminted[breaklines]{c++}{数论/斯特林.cpp}
\subsection{错排}
\inputminted[breaklines]{c++}{数论/错排.cpp}
\subsection{斐波那契-矩阵快速幂}
\inputminted[breaklines]{c++}{数论/斐波那契-矩阵快速幂.cpp}
\subsection{逆元}
\subsubsection{逆元-扩展欧几里得}
\inputminted[breaklines]{c++}{数论/逆元-扩展欧几里得.cpp}
\subsubsection{逆元-递推}
\inputminted[breaklines]{c++}{数论/逆元-递推.cpp}
\subsubsection{阶乘逆元}
\inputminted[breaklines]{c++}{数论/阶乘逆元.cpp}
\subsection{欧拉函数}
\subsubsection{欧拉函数-单独求解}
\inputminted[breaklines]{c++}{数论/欧拉函数-单独求解.cpp}
\subsubsection{欧拉函数-筛法}
\inputminted[breaklines]{c++}{数论/欧拉函数-筛法.cpp}
\subsubsection{欧拉函数-线性筛}
\inputminted[breaklines]{c++}{数论/欧拉函数-线性筛.cpp}


\section{其他}
\subsection{尼姆博弈}
\inputminted[breaklines]{c++}{其他/尼姆博弈.cpp}
\subsection{闰年}
\inputminted[breaklines]{c++}{其他/闰年.cpp}
\subsection{阶乘-万进制数组模拟}
\inputminted[breaklines]{c++}{其他/阶乘-万进制模拟.cpp}
\subsection{读写挂}
\inputminted[breaklines]{c++}{其他/读写挂.cpp}

%\newpage
%\section{Others}

\end{document}
